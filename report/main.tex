\documentclass[12pt,a4paper]{article}

% Kódovanie a jazyk
\usepackage[utf8]{inputenc}
\usepackage[T1]{fontenc}
\usepackage[slovak]{babel}

% Rozloženie strany
\usepackage{geometry}
\geometry{top=2.5cm, bottom=2.5cm, left=2.5cm, right=2.5cm}

% Ak chcete deliť projekt na subfiles   
\usepackage{subfiles}

% Balíčky pre matematiku, grafiku, tabuľky a odkazy
\usepackage{amsmath,amssymb}
\usepackage{graphicx}
\usepackage{booktabs}
\usepackage{hyperref}

% Balíček pre citácie (biblatex + biber)
\usepackage[backend=biber,style=authoryear]{biblatex}
\addbibresource{literatura.bib}
\usepackage{tocloft}


% definujeme leader pre sekcie a podsekcie
\renewcommand{\cftsecleader}{\cftdotfill{\cftdotsep}}
\renewcommand{\cftsubsecleader}{\cftdotfill{\cftdotsep}}

\begin{document}

%--- TITULNÁ STRANA ----------------------------------------------------------
\begin{titlepage}
  \centering

  % Horný riadok – fakulta a univerzita
  {\large
    Fakulta matematiky, fyziky a informatiky\\
    Univerzity Komenského, Bratislava
  }\par

  \vspace{4cm}

  % Názov práce
  {\bfseries\LARGE
    Projekt z vedy o sieťach
  }\par

  \vspace{0.5cm}

  \vfill

  % Autori v ľavom dolnom rohu
  \begin{flushleft}
    {\itshape
      Tomáš Antal\\
      Teo Pazera\\
      Andrej Špitalský \\
      3DAV
    }
  \end{flushleft}

  % Dátum v pravom dolnom rohu
  \begin{flushright}
    \today
  \end{flushright}

\end{titlepage}
%---------------------------------------------------------------------------

% Obsah a zvyšok dokumentu
\newpage

\section{Úvod}
...
\newpage

% Napríklad ďalšie časti ako samostatné subfiles:
\subfile{data.tex}
\newpage
\subfile{descriptive.tex}
\newpage

\end{document}
